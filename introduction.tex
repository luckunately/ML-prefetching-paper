\section{Introduction}
% DRAM is a bottleneck in datacenters.
Increasing memory requirements for applications, e.g., machine learning, coupled with the slowdown of DRAM scaling~\cite{dram-1, dram-2}, makes DRAM one of the costliest components in data centers, constituting as much as 30\% of the entire cost~\cite{meta}. Thus prefetching is essential to reduce the number of page faults and improve application performance.

\textbf{NEED TO ADD MORE CONTEXT HERE to explain why we need prefetching for page faults.}

% Introduction to prefetching

Prefetching is a technique used to reduce the number of page faults by predicting the pages that will be accessed in the future and fetching them into the cache before they are accessed.

% Traditional prefetching

Prefetching is a very rich field with many different approaches. Traditional prefetching algorithms use heuristics to predict the next page to prefetch. These heuristics are based on the recent access patterns of the application and guess the next page to prefetch based on these patterns using a pre-defined algorithm~\cite{Markov, Survey}. 

% Machine learning prefetching

With the rececnt bloom of machine learning (ML) techniques, researchers have started to apply ML to the problem of prefetching.

\textbf{WHAT ARE WE ACTUALLY PROPOSING?}
